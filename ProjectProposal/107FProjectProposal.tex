\documentclass[11pt]{article}

\usepackage{sectsty}
\usepackage{graphicx}

% Margins
\topmargin=-0.45in
\evensidemargin=0in
\oddsidemargin=0in
\textwidth=6.5in
\textheight=9.0in
\headsep=0.25in

\title{Project Proposal for PHYS 2150 \\ Version 1.0 -- for the TA}

\author{ Ultraviolet Catastrophe \\ Jensen Lee, Liam Ziegenhorn,Gabe Valletto }
\date{\today}

\begin{document}
\maketitle	
\pagebreak

% Optional TOC
% \tableofcontents
% \pagebreak

%--Paper--
\section{Project Definition}

Recent Announcements
SM
unread,Office Hour Time Change
Hi Folks,  I am changing my office hours today from 12-2 pm to 11 am - 1 pm.  Best,  Sara

Posted on:

Sep 8, 2025, 9:39 AM
PHYS 2150-100,101,102,103,104,105,106,107,108,109,110,111,112:Experimental Physics 2
PHYS 2150 Experimental Physics 2

Working within the given restriction in regards to topic, our group has decided to focus on a specific range of input frequrency light and how the EQE changes over time as the cell is stressed. Our goal is to see if the number of C60 usage cycle has an impact on the rate of EQE degredation.

\vspace{1em}

\subsection{The Question}

The question our group wants to consider is does the C60 usage cycle number affect the degredation of the EQE over time. specifically, we want to look at the 700 to 750 nanometer range of input light.

We have noticed in the first couple of weeks of measurement that this specific range of light wavelength often shows more degredation than more middling wavelengths. In addition to this, the EQE measured in this range sometimes seems independant of effects other regions often demonstrate. In summary, the region we are interested in does not always seem to behave similarly to the rest of the spectrum.

\vspace{1em}

\subsection{The Expectations}

The expectation is that as cell stressing and time go on, the value for EQE in the region decrease. We will not be comparing the rate of degredation of this region with other regions of the light spectrum. Additionally, the expectation is that C60 usage cycles will negatively affect EQE over time and cause the EQE to degrade faster.

\vspace{1em}

\subsection{The Plan}

Data usage:
\begin{itemize}
	\item Current data of each pixel of each cell
	\item Power data of each cell
	\item C60 usage cycles for each cell
\end{itemize}

An average for each pixel in each cell will be found on the 700 - 750 nm wavelength range and then sorted by the C60 cycles of each cell. The average value for each pixel will be averaged each week to find a value for the mean EQE for each C60 cycle set, in addition, the variance for this mean EQE will also be caluculated. At the end, the mean EQE for each C60 cycle set will be ploted and compared to draw a conclusion and trend line.

A line of best fit will then be found for each C60 cycle set the rate of change of each line will be compared. A Chi squared test will be performed to quantify how related the two sets are.

the main figure we expect from this are graphs for how each cell in a C60 cycle set changes over time and a graph of how the average EQE for each C60 cycle set changes over time while compared to the other C60 cycle sets over time.


\vspace{2em}

\section{Tasks, Roles, and a Brief Plan}

\subsection{Roles}

Because there are three people in this group, tasks are delegated to three different roles:

\begin{itemize}
	\item Data Lead: Jensen Lee
	\item Stats Lead: Gabe Valetto
	\item Presentation Lead: Liam Ziegenhorn
\end{itemize}

\textbf{Data Lead}: The data lead is to focus on the data and process the raw information that needs to be analyised.

\textbf{Stats Lead}: The role of the stats lead is to use the data processed by the data lead and create clear, digestable graphs and interpret the data in such a way that the data becomes presentable and easy to understand.

\textbf{Presentation Lead}: The Presentation lead is to agregate the data, graphs, and interpretations of the data into a presentable paper and memo.

\vspace{1em}

\subsection{A Brief 3 Week Plan}

\begin{table}[h!]
\centering
\resizebox{\columnwidth}{!}{%
\begin{tabular}{lllll}
       & Data Lead                & Stats Lead                 & Presentation Lead              &  \\
Week 1 & Finish processing data   &                            &                                &  \\
Week 2 &                          & Begin doing data analysis  &                                &  \\
Week 3 &                          &                            & work on Presentation and paper &  
\end{tabular}
}
\end{table}
* N/A is listed for the Stats and Presentation Lead because their duties depend on having processed data to analyse. Their responsibilities will come later down the line and the responsibilities of the Data lead will decrease significantly after the third week
\end{document}
